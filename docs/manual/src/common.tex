\begin{titlepage}
  \begin{center}

  {\Huge UTIL\_HELPER}

  \vspace{25mm}

  \includegraphics[width=0.90\textwidth,height=\textheight,keepaspectratio]{img/AFRL.png}

  \vspace{25mm}

  \today

  \vspace{15mm}

  {\Large Jay Convertino}

  \end{center}
\end{titlepage}

\tableofcontents

\newpage

\section{Usage}

\subsection{Introduction}

\par
Util helper are sets of different header functions that can be included with Verilog source code. These functions
are safe for synthesis if used correctly. Such as generating contants used in the source code. The current helper
utility files are.

\begin{itemize}
\item util\_helper\_math.vh
  \begin{enumerate}
    \item int clogb2(value) ... will return the log base 2 of the argument (value), rounded up to the nearest integer.
    \item int cmax(max1, max2) ... will return the number that is the max of the arguments max1, max2.
    \item int abs(value) ... will return the absolute value of the argument passed.
  \end{enumerate}
\end{itemize}

\par Fusesoc will require a dependency include, like the following.
\begin{lstlisting}[language=XML]
dep:
  depend:
    - AFRL:utility:helper:1.0.0
\end{lstlisting}

It will also have to be included in the file you would like to use the functions in. An example is...
\begin{lstlisting}[language=Verilog]
module flip_flop (input clk, input rst, input Q, output D);

`include "util_helper_math.vh"

endmodule
\end{lstlisting}

\subsection{Dependencies}

\par
The following are the dependencies of the cores.

\begin{itemize}
  \item fusesoc 2.X
  \item iverilog (simulation)
\end{itemize}

\subsection{fusesoc}
\par
Fusesoc is a system for building FPGA software without relying on the internal project management of the tool. Avoiding vendor lock in to Vivado or Quartus.
These cores, when included in a project, can be easily integrated and targets created based upon the end developer needs. The core by itself is not a part of
a system and should be integrated into a fusesoc based system. Simulations are setup to use fusesoc and are a part of its targets.

\subsection{Source Files}

\subsubsection{fusesoc\_info File List}
\begin{itemize}
\item src
	\begin{itemize}
	\item {'src/util\_helper\_math.vh': {'is\_include\_file': True}}
	\end{itemize}
\item tb
	\begin{itemize}
	\item tb/tb\_helper.v
	\end{itemize}
\end{itemize}


\subsection{Targets} \label{targets}

\subsubsection{fusesoc\_info Targets}
\begin{itemize}
\item default
	\begin{itemize}
	\item[$\space$] Info: Include all helper functions for verilog.
	\end{itemize}
\item sim
	\begin{itemize}
	\item[$\space$] Info: Test helper funtions.
	\end{itemize}
\end{itemize}


\subsection{Directory Guide}

\par
Below highlights important folders from the root of the directory.

\begin{enumerate}
  \item \textbf{docs} Contains all documentation related to this project.
    \begin{itemize}
      \item \textbf{manual} Contains user manual and github page that are generated from the latex sources.
    \end{itemize}
  \item \textbf{src} Contains source files for util\_helper.
  \item \textbf{tb} Contains test bench files.
\end{enumerate}

\newpage

\section{Simulation}
\par
A barebones test bench for iverilog is included in tb/tb\_helper.v . This can be run from fusesoc with the following.
\begin{lstlisting}[language=bash]
$ fusesoc run --target=sim AFRL:utility:helper:1.0.0
\end{lstlisting}

\newpage

\section{Module Documentation} \label{Module Documentation}

\par
This project has multiple functions.

\begin{itemize}
\item \textbf{util\_helper\_math.vh}
\item \textbf{tb\_helper.v}
\end{itemize}
The next sections documents various functions per header.

